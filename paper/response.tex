\documentclass[11pt]{amsart}
\usepackage{graphicx,amssymb,amsthm,setspace,fullpage,natbib}
\onehalfspacing

\title{Random Union of Gametes}
\author{Mason Liang}
\date{\today}

\begin{document}
\textit{Responses in italics}
\section{Reviewer 1}
The authors present a model for understanding how moments of admixture
proportions in a sample of individuals from a population change over time.
Unlike earlier methods that allow inference of admixture dates from ancestry
segment lengths, their method should be applicable in situations where local
ancestry estimates cannot be reliably obtained. In contrast to Verdu and
Rosenberg 2011, who focus on the admixture fraction for a random individual at
one locus, Liang and Nielsen study the mean admixture fraction of a haploid
individual’s genome, which makes their results both more intuitive to me, but
also directly interpretable in comparison to sample population data.
[Specifically, I find their definition of H quite meaningful, whereas I could
never figure out how $H_{1,g}$ in the earlier work could take on values
different from 0, 1/2, or 1, since it describes the admixture proportions of one
individual at exactly one locus, but that may be my own inflexible
understanding.] But furthermore they incorporate drift and the random
recombination Aside from the numerous typos, in both text and equations, I have
observed no substantial problems with the modeling, and find the applications to
African American data well-chosen. It would have been neat to see the method
applied to an older admixture event for which local ancestry estimates are not
possible, but I understand it would be difficult to validate such an inference
and older admixed datasets might not be readily available. Overall, I think this
is an interesting and relevant work.\\
line 57 - delete second “model”
	\textit{This has been deleted.}\\
lines 116-117 equations: should the $M$ rather be $G$ in the equations?
	\textit{Yes, we have changed them to G's}\\
line 118 - ``factions" $\rightarrow$ ``fractions"
	\textit{We have added the r.}\\
line 120 - ``to the left"
	\textit{We have added the missing ``the''}\\
line 159 - is the location of the sites in genetic distance?
	\textit{, this clarification has been added to the manuscript.}\\
line 166 - delete second comma at end of equation
	\textit{We have deleted it.}\\
line 175 - ``interference"
	\textit{We have corrected the spelling.}\\
line 181 - ``this assumption" could you clarify which assumption (no crossover
interference?)
	\textit{Yes, the line now reads ``However, none of the mathematical results of
	this paper will require this assumption of no crossover interference.''}\\
line 210 - I would have appreciated some explanation or definition of
$\kappa_n$.
	\textbf{We now define $\kappa_n$ on line LINE NUMBER GOES HERE.}\\
line 229 - ``these are similar"
	\textit{We have added the missing ``are''.}\\
line 249 - ``becuase"
	\textit{We have corrected this.}\\
line 262 - do you define which population you infer to admix into the other? I
assume you infer the migration rate of Europeans into an African population, but
 this is not clear from the text.
	\textit{We have clarified the model ``We used a 3-parameter model of constant
	admixture in which the admixed population is entirely of African ancestry
	before generation $g_start$. For $g_{start}\leq g\leq g_{stop}$, the gene-flow
	probability from the European source population is $s_g=s$, with $s_g=0$
	otherwise. ''}\\ line 263 - Did you explain which values you used in eqn 1?
	Ie, I assume you needed to choose $L$? Or if not, could you explain?
	\textit{We used $R=33$. This has been included in the text.}\\
line 291 - ``European and African"
	\text{This has been fixed.}\\
lines 293 and 295 – ``g\_stop" $\rightarrow$ ``$g_{stop}$"
	\textit{These have been fixed.}\\
line 294 - It is possible that these individuals may choose not to self-identify
as African American, so an alternate explanation is that individuals with more
recent admixture may not be in this self-identified population. \\
	\textit{This is an explanation we had not considered. We have added it.
	``Another possible explanation for this estimate is that individuals with more
	recent admixture tend to not self-identify, or were otherwise not included in
	the ASW population. The effects of this sampling bias would be the same as the
	decline in gene flow we observed.''}\\
Figure 2 - I would like this figure even more if it had estimates of $k_1$ and
$k_2$ for each simulated population.\\
	\textit{Figure 2 has been updated to include the sample variance for each
	replicate. For simplicity, we did not include the sample mean for simplicity
	because it is more easily estimated by eye.}\\
Figure 3 - ``but the two are very similar"\\
	\textit{We have added the missing ``re''.}


\section{Review 2} This article proposes theoretical derivations for moments of
the distribution of ancestry proportions in admixed populations. This is an
interesting topic, and the authors propose a general recursive method to
calculate high-order moments of the distribution that was not presented before.
The manuscript takes as a starting point the model of Verdu and Rosenberg (VR)
for the estimation of variance in admixed populations. VR does not take into
account drift or the finiteness of the genome, and the present manuscript sets
out to model these effects. The authors also distinguish between expectations
within population (i.e., given a fixed population genealogy), and between
populations (i.e., between individuals having independently sampled
genealogies), and propose an inference scheme for structure data.

Even though this is an interesting manuscript, a few issues should be
addressed.\\

\subsection{} Gravel (2012) also discusses mean and variance in ancestry
proportion in models with finite genome, drift, and continuous migrations,
although not all three jointly (pp 611-613 and 618-619). It also presents an
application to admixture timing in the ASW, also finding that variance in
admixture proportion supports more than a single pulse of admixture. The method
was also used in Botigue et al 10.1073/pnas.1306223110. This manuscript is more
detailed than Gravel (2012) on this topic, but there is considerable overlap. At
the very least, the authors must acknowledge the overlap and compare the results
in some detail.\\
\subsection{} The equation at line 176.5 is comparable to the second term in
equation (8) in Gravel (2012). However, the first term of equation (8) (dominant
in many practical cases) appears to be missing. Even though the present
manuscript takes into account correlations among individuals induced by the
genealogy, it does not seem to take into account the correlation among loci
induced by genealogies. If I understand things correctly, the authors start off
with a "Diploid Wright-Fisher" model, but then perform calculations according to
a "Markovian Wright-Fisher" model (see Figure 2B in Gravel (2012)). If I *don't*
understand things correctly, I suppose that clarifications about exactly what
expectations are calculated would be useful (see below).

Fortunately, even if I am correct, I believe that the discrepancy can be
corrected in the present models using the law of total variance. At line 245.5,
the authors do something similar, but condition the variance over "population"
genealogies. They could solve most of this problem by conditioning over
"individual" genealogies, as in Gravel (2012).
	\textit{ We do not use a "Markovian Wright-Fisher" model in this paper. The
	equation in this manuscript most comparable to equation (8) in Gravel (2012)
	is equation XX. Equation YY in our manuscript gives the expectation of the
	second $k$-statistic, in common usage, this would be the variance, but as we
	explain in lines .... in our case, it is not. The variance of the admixture
	fractions (marginalizing over all genealogies) is computed in equation XX.
	This does have two terms corresponding to the two in  equation (8) in Gravel
	(2012). We have adopted the clearer terminology from that paper and now refer
	to the two as the "assortment variance" and "genealogy variance". }

\subsection{} The manuscript presents a moments method to estimate parameters
directly from the distribution of inferred admixture proportions. Even though
distributions of ancestry proportions are indeed informative of demography, its
higher moments are also very sensitive to a diversity of effects that are not
accounted for here, such as assortative mating, population structure, nonrandom
sampling, sex-biased admixture, and correlated errors in ancestry assignments.
If this method is to be considered for practical inference, these caveats should
be presented clearly to avoid mis-use. I have found practical application of
variance estimates to be perilous in most practical cases.
	\textit{ We agree that these of deviations from the Wright-Fisher model and
	our model of admixture fraction estimation  will impact our ability to make
	accurate inferences. Assortative mating, population structure, would tend to
	increase $\mathbb{E}[k_2]$, which would bias our estimates of the admixture
	time downwards. Correlated errors in ancestry assignment, on the other hand,
	would cause $k_2$ to be downwardly-biased, which would make our estimate of
	the admixture time too big. }

\subsection{} By contrast, I suspect that drift and within-vs-cross-population
effects will be small to negligible except for very small populations. This is
not to say that it is not useful to model these, but it may be useful to point
this out. The authors show in figure 2 important differences in the mean
ancestry proportion after 500 generations in an admixed population, but I found
little detail as to how the simulation was conducted. Gravel (2012) estimates
the standard deviation in ancestry proportion across replicate populations at
0.05 for a population of size 100, so I'm assuming that the simulation presented
here uses a population of similarly small size.
	\textit{ We have included additional details about the simulations. The
	simulations used a population size of 2N=1000, with m=0.5 and total
	recombination distance of 1 Morgan. We have added these details to the paper.
	The results of our simulations are similar to those of Gravel (2012) and
	Rosenberg (????) when the admixture pulse is recent. As $T$ becomes larger,
	the effects of genetic drift become significant, and our simulation results
	diverge. }

\subsection{}
Finally, if simulations have been performed, it would be a good idea to provide comparisons with the
theoretical results!
[Comment: I have simulations, I just need to make the plots.]


\subsection*{Minor points/clarifications}\ \\
Line 116.5: M is not defined. It may also be useful to spell out "the expectations are taken over ..."--I'm not
sure I understand exactly what was done. My understanding is that M is the population genealogy (which
yields a given distribution G), and that in the LHS terms have expectations taken over both populations and
individuals, whereas the RHS is taken over individuals in a fixed population.\\
Line 134-136
It could be good to clarify which expectations are used here as well, and what terms are being estimated.
For example, E[k1] corresponds to the LHS term at line 116.5 . However, E[k2] is an estimator for the RHS
term, since we are taking a sample mean and not the expected mean. It may be good to be explicit.\\
Line 144.5
It'd be useful to point the reader to the appendix for the definitions of the v's.\\
Line 249 "because"\\
\\
I don't always sign my reviews, but I gave so many hints at this point that I might as well--\\
Simon Gravel

\section{Reviewer 3}
Liang and Nielsen consider the joint distribution of admixture proportions
$H_{i(g)}$, $i\in 1,\ldots, n$ of n individuals sampled from an admixed
population g generations after the start of admixture. Previous work apparently
explored the marginal distribution of the admixture proportion of a single
individual drawn from an admixed population. (This distribution was marginal
both in the sense of applying, marginally, to a single individual, and also
being marginalized over all possible genealogies). By contrast, Liang and
Nielsen explore the joint distribution of H's conditional on the unknown
genealogy connecting the samples.

They derive expressions for the expected values of various k-statistics that can
be computed from a sample of H's, and then, by relating these expected values to
admixture parameters in some special cases they develop a method of moments
estimator for these admixture parameters.

This is an interesting effort to relate admixture proportions, which are
routinely estimated by programs like structure (they are the individual q
values), to the parameters of mechanistic models of population admixture and
could provide a valuable tool for increasing the interpretability of results
from structure.

Below I provide a few comments that I hope will be helpful.

\subsection{}
I found the author's implicit invocation of de Finetti's representation theorem
and its extensions by Hamos and Savage a little hard to wrap my head around
here, primarily because it is not quite clear to me what they are considering
$\mathcal{G}$ to be. At lines 104 and 105 they seem to be making the case that
conditional on a certain realization of a random distribution the H's will be
iid. But, then, later, they seem to be equating $\mathcal{G}$ with a realization
from the distribution of possible genealogies. I guess the confusing part for me
here is that if one is trying to compute the expected value of $k_2$, then
presumably you are conditioning on a certain realization of $\mathcal{G}$, and
if that is the case and you are adhering to the interpretation of $\mathcal{G}$
as being a realization from a random distribution a la the representation
theorem, then $H_{i(g)}$ and $H_{j(g)}$ should be conditionally independent and
hence the terms involving the expectations of their products would simplify
quite a bit further than what we find above line 137. So, on the one hand the
authors are saying that the H's are conditionally iid, and on the other they are
saying that they are correlated and that is central to their derivation. It
would be great if this could be clarified somewhat.\\
\textit{
	Some of this confusion may arise from an error on our part in which we wrote
	$\mathcal{M}$ when we meant to write $\mathcal{G}$. We apologize for this. In
	this paper, we take $\mathcal{G}$ to be the genealogy of the admixed
	population. Strictly speaking, this genealogy is more informative than the
	limiting ``empirical measure'' because many of the genealogical ancestors are
	not genetic ancestors, and so do not influence the observed admixture
	fractions. All of the expectations we compute in this paper (unless explicitly
	stated to be conditional), are over the distribution of $\mathcal{G}$. When
	marginalized over this random distribution, $H_{i(g)}$ and $H_{j(g)}$ are no
	longer independent, so computing $\mathbb{E}[k_i]$ becomes non-trivial.
}

\subsection{}
One question I have as I read this is, "How far from sufficient are the H's?"
The Hs themselves are going to be functions of the underlying "cluster of
origin" process estimate by, say, structure, when using the linkage model. So,
instead of boiling everything down to a single real number between 0 and 1 for
each individual (the H's), one might consider trying to do inference from, say,
the distribution of the lengths of tracts of common descent from different
clusters. I suspect that might carry considerably more information that is
relevant to estimating admixture-model parameters than just the H's. How much
information is lost by just considering the H's? Some discussion of this would
be appreciated.\\
\textit{
	The problem on inferring admixture times from the distribution of admixture
	tract lengths has been considered in papers such as Pool and Nielsen 2009,
	Gravel 2012, and Liang and Nielsen 2014. We see two key distinctions between
	using admixture fractions and admixture tracts. The first is that admixture
	fractions may be available in situations where admixture tracts are not, such
	as when marker density is low or a recombination map is not available. Even
	when both are available, estimating admixture fractions is more robust than
	estimating admixture tract length distributions. The second is that finding
	the theoretical admixture tract length distribution is non-trivial. The
	mixture-of-exponentials approximation derived in Gravel 2012 is not always
	accurate.
}

\subsection{} Given 2, it seems that this method would be particularly
appropriate in cases where a map is not available --- i.e. in molecular ecology
contexts studying non-model organisms. This being the case, it would be good if
the authors were able to address how robust (or not) the model might be to
certain departures from the sampling model and the mechanistic admixture model
they employ. Of most concern to me as regards the application of their method in
molecular ecology contexts is the effect of population structure within the
admixed population. Population structure within the admixed population will
clearly alter the observed distribution of the H's. I think it would be good to
know how the strength of an effect like population structure compares to the
strength of the admixture parameters in their model in altering the distribution
of the H's. I suspect that it could have a large effect that could complicate
estimation of the admixture parameters in populations with unknown spatial
gradients of introgression, etc.\\
\textit{
}

\subsection{}
Echoing 3, a full set of simulations evaluating the power of the
method across a range of scenarios (in terms of number of markers and the
differentiation between the two admixing populations) would be nice to have, as
well as assessments of the robustness (or not) of the model to population
structure within the admixed population. In some ways, it appears to me that the
estimation of these sorts of admixture parameters could be analogous to the
estimation of K---the number of clusters---using the program structure, an
exercise that is fraught with peril and warrants caution because real
populations seldom behave like idealized ones. It would be nice if the caveats
involved in this method could be stated up front. (not that this would
necessarily keep molecular ecologists from abusing it somewhat...the estimation
of K being a case in point).\\
\textit{
}

\subsection{}
The assumption that the estimation error of $H_{i(g)}$ is
independent of $H_{i(g)}$ at line 208 seems like it might not be the best
assumption. A more reasonable assumption might be that the variance of
$\epsilon_i$ is proportional to $H_i(1-H_i)$, as is common when dealing with
proportions. Of course, this complicates the calculations. Perhaps some type of
variance-stabilzing transformation (like arc-sine square root) might be
useful.\\
\textit{
	TBD
}

\subsection{}
It would be nice to have a github repository with the all the
scripts the authors used so that their analyses are easily reproducible. In
fact, it seems like some simple software (in R) for example, implementing the
authors' method is conspicuously absent. Many journals today require that data
sets be archived to insure reproducibility, and I think it is incumbent upon
theoreticians, likewise, to offer clean software or scripts to ensure
reproducibility, and in some cases ease interpretation of methods.\\
\textit{The
simulation code and scripts to reproduce the plots in the manuscript have been
uploaded to github:
\texttt{https://github.com/z2trillion/local\_ancestry\_moments}.}

\subsection*{Minor points}
\subsubsection{}
Equation above line 117. Should the
$\mathcal{M}$ be $\mathcal{G}?$ If not, $\mathcal{M}$ needs definition.\\
\textit{
	We have changed the $\mathcal{M}$ to a $\mathcal{G}$.
}

\subsubsection{}
I think that the second summation in the equation below line 136 should be over
$i\neq j$, rather than $i,j = 1$ up to n.\\
\textit{
	Our original version was wrong. We have made this change.
}

\subsubsection{}
Line 148: Wait a moment (what a terrible pun!). Isn't $k_2$ estimating a central
moment? Perhaps some more general discussion of $k$-statistics earlier in the
paper would be good. \\
\textit{
	The $k$-statistics are estimators for cumulants: $\mathbb{E}(k_i) = \kappa_i$
	while the $h$-statistics are estimators for central moments: $\mathbb{E}(h_i)
	= \mu_i$. For $i=1,2,3$, the central moment and cumulant are the same, with
	$h_i$ and $k_i$ having the same formula, but for $i\geq 4$, the $i^\text{th}$
	cumulant is not equal to the $i^\text{th}$ central moment. In the revised
	paper, we explain this difference in lines TODO(mason).
}

\subsubsection{} The manuscript would benefit from a few extra rounds of
proofreading.\\
\textit{
	We have gone through a couple more times.
}
\end{document}
